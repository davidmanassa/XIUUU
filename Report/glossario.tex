

\chapter{Glossário}
\paragraph{}

\newglossaryentry{\textbf{ Agente de Confiança}{ Toma o papel central e sabe todas as chaves de sessão geradas entre quaisquer dois intervenientes, e não as partilha com mais ninguém.}

\newglossaryentry{\textbf{ Alice}}{ Agente emissor da mensagem.}

\newglossaryentry{\textbf{ \textit{Bits}}}{ É a menor unidade de informação que pode ser armazenada e transmitida através de uma comunicação de dados, este pode ser apenas 0 ou 1.}

\newglossaryentry{\textbf{ \textit{Bytes}}}{ É uma unidade de informação digital, que é equivalente a oito bits.}

\newglossaryentry{\textbf{ Bob}} { Agente recetor da mensagem.}

\newglossaryentry{\textbf{ \textit{Chat}}}
{
  Troca de mensagens entre dois utilizadores em tempo real.
}

\newglossaryentry{\textbf{ Chave Privada}}
{
Chave em que só o dono pode ter acesso.
  
}

\newglossaryentry{\textbf{ Chave Pública}}{
  Chave que pode ser conhecida por todos.
}

\newglossaryentry{\textbf{ Chave Secreta}}
{
  Igual à chave privada.
}

\newglossaryentry{\textbf{ Chave Simétrica}}
{
  Chave que é igual para os dois agentes da troca de mensagens.
}

\newglossaryentry{\textbf{ Claire}}
{
  Agente que ataca as comunicações.
}

\newglossaryentry{\textbf{ Criptograma}}
{
  É um texto cifrado que obedece a um código e a uma lógica pré-determinada para decifrar a mensagem.
}

\newglossaryentry{\textbf{ \textit{Hash}}}
{
  Valor numérico de comprimento fixo que identifica exclusivamente dados.
}

\newglossaryentry{\textbf{\textit{ Help}}}
{
  No contexto da aplicação, um \emph{help} é uma ajuda de modo textual que resume o modo de funcionamento do programa para o utilizador.
}

\newglossaryentry{\textbf{ Interface}}
{
  Ferramenta que permite que o utilizador comunique com as diferentes partes de um programa.
}

\newglossaryentry{\textbf{ \textit{JAVA}}}
{
  É uma linguagem de programação orientada a objetos, desenvolvida pela \textit{Sun Microsystems} na década de 90. Hoje pertence à empresa Oracle.
}

\newglossaryentry{\textbf{ \textit{Localhost}}}
{ 
Localização do sistema que está a ser usado.
}


\newglossaryentry{\textbf{ NetBeans}}
{
  É uma plataforma de desenvolvimento de software gratuita. Esta destina-se principalmente ao desenvolvimento de código Java, mas também suporta outras linguagens.
}
\makeglossaries