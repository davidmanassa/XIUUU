\chapter{Introdução}
\label{chap:intro}

\section{Enquadramento}
\label{sec:amb} 
Este projeto, denominado \textit{XIUUU: Troca de Segredos Criptográficos Seguro
}, foi realizado no contexto da unidade curricular de Segurança Informática, que se enquadra no terceiro ano de Licenciatura em Engenharia Informática da Universidade da Beira Interior, no ano letivo 2019/2020.


\section{Motivação}
\label{sec:mot}
O  projeto  foi  proposto,  e  será  avaliado,  pelo docente  Pedro  Inácio, e tem em vista a implementação e aprofundamento dos conhecimentos adquiridos durante as aulas práticas e teóricas da unidade curricular em que se enquadra, ao longo do atual semestre.

\section{Objetivos}
\label{sec:obj}
Este projeto tem como base a implementação de um sistema que permita a troca de segredos entre duas entidades: cliente e servidor. Possui diversas funcionalidades implementadas através de conceitos das áreas de criptografia.

\section{Constituição do grupo}
\label{sec:const} 
A constituição do grupo de trabalho a que se deve a realização deste documento, e respetivo projeto, foi da responsabilidades dos próprios elementos, que se encontram aqui descritos:
\begin{itemize}
    \item Ana Maria Delgado, a37668
    \item David Pires, a37272   
    \item Gabriel Esteves, a38488
    \item Inês Roque, a37174 
    \item Renato Lopes, a37408
\end{itemize}

\section{Organização do Documento}
\label{sec:organ}

De modo a refletir o trabalho que foi feito, este documento encontra-se estruturado da seguinte forma:
\begin{enumerate}
    \item Introdução - nesta secção será feita  uma descrição geral do projeto, identificando os objetivos, a sua finalidade e enquadramento.
    \item Estado da Arte - apresentação de várias definições importantes para a compreensão do documento. São também descritas as tecnologias utilizadas para elaborar o projeto.
    \item Engenharia de Software - apresentação de diagramas que ajudarão na compreensão do projeto.
    \item Implementação - descrição mais aprofundada do programa desenvolvido e é feita uma listagem de um conjunto estruturado de testes.
    \item Reflexão Critica e Problemas encontrados - análise ao trabalho realizado e são comparados os objetivos iniciais com os atingidos. É também feita uma revisão da segurança do sistema implementado.
    \item Conclusão - Revisão do documento e funcionalidades a implementar caso o projeto continuasse a ser desenvolvido fora do seu contexto atual.
\end{enumerate}
