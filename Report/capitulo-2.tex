\chapter{Estado da Arte}
% OU \chapter{Trabalhos Relacionados}
% OU \chapter{Engenharia de Software}
% OU \chapter{Tecnologias e Ferramentas Utilizadas}
\label{chap:estado-da-arte}

\setlength{\parskip}{1em}

\section{Introdução}
\label{chap2:sec:intro}
O projeto foi desenvolvido para computador, utilizando a linguagem de programação \textit{JAVA}, compilado e testado com o IDE \emph{NetBeans} em sistemas operativos \emph{Windows} ou \emph{Linux}. A aplicação corre em Graphical User Interface (GUI). O relatório foi desenvolvido utilizando a ferramenta \emph{Share}\LaTeX.
Nesta secção são abordados tópicos e definições relevantes para compreender alguns algoritmos/protocolos utilizados neste projeto, nomeadamente: o algoritmo \textit{Password Based Key Derivation Function 2} (PBKDF2), o protocolo de acordo de chaves \textit{Diffie-Hellman, Puzzles de Merkle, o Rivest, Shamir e Adleman} (RSA),  \par

No final do capítulo é ainda descrita a linguagem de programação utilizada e a razão da sua escolha para o desenvolvimento deste projeto.

\section{Conceitos Relevantes}
\label{chap2:sec:...}

\subsection{\textit{Password Based Key Derivation Function 2}(PBKDF2)}
Esta função vai cifrar um segredo com o auxílio de uma palavra-passe, onde o cliente pode escolher entre várias funções de \textit{hash}. \par
Com a palavra-passe introduzida, o programa irá pegar na mesma, em \textit{salt}, no número de iterações que terá de executar e no tamanho da chave. O \textit{salt} em conjunto com a palavra-passe utilizada irá gerar um novo vetor de \textit{bytes}. Através disso e com a chave secreta que foi gerada com o auxílio da função de \textit{hash}, vai calcular uma nova chave (com o algoritmo AES). Após isso, usa a chave gerada anteriormente e a mensagem, encriptando a mesma com "AES/CBC/PKCS5Padding"usando o vetor de inicialização e transforma este conjunto de \textit{ bytes} em \textit{base64} para assim ser percetível ao utilizador.

\subsection{Protocolo de acordo de chaves \textit{Diffie-Hellman}}
Este algoritmo serve para  partilhar um segredo sem que tenha existido uma troca de chaves prévia entre a Alice (cliente emissor) e o Bob (cliente recetor). Usando o protocolo de acordo de chaves Diffie-Hellman um adversário (ou Claire) pode escutar a comunicação, mas não pode manipulá-la.   
\paragraph{}
Este acordo de chaves descreve-se da seguinte forma:
\begin{itemize}
    \item A Alice e o Bob escolhem um número primo fixo suficiente grande (p). Este número  pode ser tornado público. No nosso caso esse número tem 1024 \textit{bits}.
    \item De seguida, é definido um gerador g entre 1 e p que pode ser tornado publico.
    \item Quando estes dois valores (p e g) são combinados entre ambos, a Alice escolhe um numero x entre 1 e p, que irá ser a sua chave secreta, e calcula $X=g^x \bmod p$ e envia X ao Bob. X pode ser público.
    \item O Bob recebe X e  escolhe um número y , que irá ser a sua chave secreta, entre 1 e p e calcula $Y=g^y \bmod p$ e envia Y para a Alice, o valor Y pode ser tornado público.
    \item Quando a Alice tiver o Y calcula k da seguinte forma: $k= Y^x \bmod p$. k será a chave secreta que ambos vão ter. O Bob calcula k recorrendo a $k=X^y\bmod p$.
\end{itemize}
   \paragraph{}
   De seguida, é feita uma assinatura deste ficheiro com SHA256 e RSA que será confirmada pelo utilizador recetor do segredo e, caso esta assinatura seja verificada, o segredo será mostrado. Caso contrário o segredo não será exibido. Esta assinatura que é efetuada pela a Alice é feita com a sua chave privada, e para que seja verificada pelo Bob ele tem que pedir à Alice a chave pública.

\subsection{Assinatura Digital}

Este mecanismo criptográfico procura substituir as assinaturas físicas tradicionais com recurso às três ferramentas criptográficas seguintes: infraestrutura de chave pública, cifra de chave pública \textit{Rivest-Shamir-Adleman (RSA)} e funções de \textit{hash}. \par

O processo típico envolve a obtenção do valor de \textit{hash} do ficheiro a assinar e a encriptação do mesmo usando a cifra \textit{RSA}. \par
Note-se que a encriptação com chave secreta não oferece qualquer ato de cifragem, já que qualquer pessoa pode desencriptar o criptograma com a chave pública da pessoa. Isto permite, contudo, que qualquer pessoa possa verificar a assinatura digital de alguém, ao contrário do que acontece com a assinatura tradicional. \par
Garantidas estão as seguintes propriedades: \par
\begin{enumerate}
    \item Integridade do texto-limpo
    \item Autenticidade da informação
    \item Não repúdio
    \item Dificuldade de falsificação
\end{enumerate}

\subsection{Funções de \textit{hash}}
Uma função de \textit{hash} é uma função que recebe um grupo de caracteres e gera um número de tamanho arbitrariamente grande, mas finito, denominado de valor de \textit{hash}. \par
Funções de \textit{hash} têm diversas funcionalidades, como por exemplo, são usadas para  aumentar a rapidez de extração de dado de uma base de dados, como também são usadas para encriptar ou desencriptar assinaturas digitais. \par
Algumas das propriedades destas funções incluem:
\begin{enumerate}
    \item Resistência à descoberta de texto original
    \item Resistência à descoberta de um segundo texto
    \item Resistência à colisão
\end{enumerate}
Para o contexto deste trabalho, as funções de hash que forma implementadas são as seguintes:
\begin{enumerate}
    \item\textit{ Message Digest 5} (MD5)
    \item \textit{Secure Hash Algorithm 256} (SHA256)
\end{enumerate}




 \par

\section{Tecnologias Utilizadas}

\subsection{\textit{Java}}
\textit{Java} é uma linguagem de programação orientada a objetos que começou a ser criada em 1991, na Sun Microsystems. Teve inicio com o \textit{Green Project}, no qual os mentores foram Patrick Naughton, Mike Sheridan, e James Gosling. Este projeto não tinha intenção de criar uma linguagem de programação, mais sim de antecipar a “próxima onda” que aconteceria na área da informática e programação. Os idealizadores do projeto acreditavam que em pouco tempo os aparelhos domésticos e os computadores teriam uma ligação.

\section{Conclusões}
\label{chap2:sec:concs}

Com esta secção ficam introduzidos os conceitos técnicos utilizados ao longo destes relatório, facilitando assim o acompanhamento do raciocínio descrito.

