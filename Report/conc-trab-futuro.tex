\chapter{Conclusões e Trabalho Futuro}
\label{chap:conc-trab-futuro}

\section{Conclusões Principais}
\label{sec:conc-princ}

A realização desta aplicação foi uma experiência positiva, apesar de não termos conseguido alcançar todos os objetivos pretendidos no enunciado do trabalho. Ao fazer o projeto, adquirimos também mais conhecimento na área da Segurança Informática. 

\section{Trabalho Futuro}
\label{sec:trab-futuro}

Achamos que, no futuro, implementar um sistema de \emph{chat}(com 2 ou mais participantes), ao invés da troca singular de segredos.
\newline Outro objetivo futuro seria fazer com que as comunicações não fossem feitas apenas no mesmo computador mas sim entre 2 ou mais computadores.
\newline Como já foi citado, não conseguimos concluir alguns dos objetivos propostos pelo Professor e por isso outro objetivo seria termina-los. São eles: usar certificados digitais X.509 nas trocas de segredos que recorrem ao RSA,implementar uma infraestrutura de chave pública para o sistema e validar cadeias de certificados nas trocas de segredos que recorrem ao RSA, pensar numa forma correta de fornecer certificados digitais a utilizadores, implementar mecanismos de assinatura digital para verificação de integridade em trocas de chave efémeras usando o \textit{Diffie-Hellman} e distribuição de novas chaves de cifra usando um agente de confiança(neste caso, a aplicação desenvolvida deve permitir que uma das instâncias possa ser configurada como agente de confiança).
\newline Em suma, demonstramos satisfação com o resultado alcançado e pensamos que, se o desenvolvimento do projeto fosse continuado este teria aplicação no dia à dia.




